\documentclass{article}
\usepackage{fullpage}
\usepackage{amsmath}

\newcommand{\q}{\quad\quad}
\newcommand{\gp}[1]{{\left({#1}\right)}}
\newcommand{\z}{\mathbf{0}}
\newcommand{\tr}{\mbox{tr}}
\newcommand{\uu}{\mathbf{u}}
\newcommand{\II}{\mathbf{I}}
\renewcommand{\gg}{\mathbf{g}}
\renewcommand{\uu}{\mathbf{u}}
\newcommand{\nn}{\mathbf{n}}
\renewcommand{\tt}{\mathbf{t}}
%\newcommand{\ta}{\mbox{\boldmath{\ensuremath{\tau}}}}
\newcommand{\ta}{\pmb{\tau}}
\newcommand{\mx}[1]{\begin{pmatrix} #1 \end{pmatrix}}
\newcommand{\px}[2]{\frac{\partial #1}{\partial #2}}

\begin{document}

\section{Navier-Stokes Equations}

$$ \rho \uu_t + \rho (\uu \cdot \nabla) \uu + \nabla p = \nabla\cdot \ta + \rho \gg $$

$$ \ta = \mu (\nabla \uu + \nabla \uu^T) $$

$$ \left[ \mx{\nn & \tt}^T (p \II - \ta) \nn \right] = \mx{\sigma \kappa \\ 0} $$

$$ \nabla \cdot \uu = 0 $$

\section{Poiseuille Flow}

Poisseuille Flow describes steady state flow through a long, narrow channel.  We assume no-slip boundary conditions on the channel walls, with the additional assumption that the channel
has infinite length. We assume the flow to be parallel to the channel and laminar, so that $u = 0$.  We also assume translational symmetry along the channel, so that $\px{v}{y}=0$.
Since we will introduce an interface with a jump discontinuity across it, it will be convenient to treat boundary conditions later.  With these assumptions, we have
$$ \uu = \mx{0 \\ v} \q \nabla\uu = \mx{0 & 0 \\ v_x & 0} \q \ta = \mu v_x \mx{0 & 1 \\ 1 & 0} \q \uu_t = \z$$
$$ (\uu \cdot \nabla) \uu = \nabla\uu \uu = \mx{0 & 0 \\ v_x & 0} \mx{0 \\ v} = \z \q \nabla \cdot \uu = \tr\mx{0 & 0 \\ v_x & 0} = 0 $$
$$ \nabla p = \mx{p_x \\ p_y} \q \nabla\cdot \ta = \mx{0 \\ \mu v_{xx}} \q \rho \gg = \mx{0 \\ -\rho g} $$
$$ \mx{p_x \\ p_y} = \mx{0 \\ \mu v_{xx} - \rho g} \q p_x = 0 \q \mu v_{xx} = p_y + \rho g $$
$$ \nn = \mx{1 \\ 0} \q \tt = \mx{0 \\ 1} $$
We note that there are many solutions to the last equation, since there could be a pressure gradient forcing the fluid to flow in addition to gravity.  We assume that there is no such
pressure gradient, so that $p_y = 0$.  This implies that pressure is constant, and it leads to
$$ v = \frac{\rho g}{2 \mu} x^2 + a x + b, $$
for some constants $a$ and $b$.

Next, consider that the left wall is at $x = -w$, the interface is at $x = 0$, and the right wall is at $x = w$.  The velocity to the left of the interface is $v^-$ with density $\rho^-$
and viscosity $\mu^-$.  Similarly, $v^+$, $\rho^+$, and $\mu^+$ describe the fluid to the right of the interface.  We assume density and viscosity are piecewise constant.  Assuming that
a velocity of $v^0$ is obtained at the interface, the boundary conditions lead to
$$ v^- = \frac{\rho^- g}{2 \mu^-} x^2 + \gp{\frac{\rho^- g}{2 \mu^-} w + \frac{v^0}{w}} x + v^0 \q v^+ = \frac{\rho^+ g}{2 \mu^+} x^2 - \gp{\frac{\rho^+ g}{2 \mu^+} w + \frac{v^0}{w}} x
+ v^0 $$
To find $v^0$, we need to consider jump conditions.
$$ \mx{\sigma \kappa \\ 0} = \z \q \left[ \mx{\nn & \tt}^T (p \II - \ta) \nn \right] = [p] \mx{1 \\ 0}  - [\mu v_x] \mx{0 \\ 1} \q [p] = 0 \q [\mu v_x] = 0 $$
$$ 0 = [\mu v_x] = \mu^+ \gp{-\frac{\rho^+ g}{2 \mu^+} w - \frac{v^0}{w}} - \mu^- \gp{\frac{\rho^- g}{2 \mu^-} w + \frac{v^0}{w}} = 
-\frac{g w (\rho^+ + \rho^-)}{2} - \frac{v^0 (\mu^+ + \mu^-)}{w} $$
$$ v^0  = -\frac{g w^2 (\rho^+ + \rho^-)}{2 (\mu^+ + \mu^-)} $$

\section{Couette Flow}

Couette Flow describes steady state shearing flow through a long, narrow channel.  We assume no-slip forced boundary conditions on the channel walls, with the additional assumption that
the channel has infinite length. We assume the flow to be parallel to the channel and laminar, so that $u = 0$.  We also assume translational symmetry along the channel, so that
$\px{v}{y}=0$.  Since we will introduce an interface with a jump discontinuity across it, it will be convenient to treat boundary conditions later.  This case differs from Poiseuille
Flow in that $g = 0$, so that $v = a x + b$.  With an interface in the middle of the channel, the new boundary conditions are $v = v_L$ at $x = -w$ and $v=v_R$ at $x = w$.
$$ v^+ = \frac{v_R - v^0}{w} x + v^0 \q v^- = -\frac{v_L - v^0}{w} x + v^0 $$
The jump conditions are the same as before, so
$$ \mu^+ \frac{v_R - v^0}{w} + \mu^- \frac{v_L - v^0}{w} = 0 \q v^0 = \frac{\mu^- v_L + \mu^+ v_R}{\mu^+ + \mu^-} $$

\section{Circular Couette Flow}

Circular Couette Flow describes the circular steady state shearing flow around an annulus, which we center at the origin.  We again assume no-slip forced boundary conditions along the boundary of the annulus.  We assume perfectly circular flow, so that $u \cdot \hat{r} = 0$, where $\hat{r}$ denotes the unit vector in the radial (away from the origin) direction; and so that the flow is rotationally invariant.  Thus, $\uu$ takes the form $\gamma \mx{-y \\ x}$, where $\gamma = \gamma \left( \sqrt{x^2 + y^2} \right) = \gamma(r)$ is a scalar function depending only on the distance from the origin (one can easily check that any such $\uu$ satisfies the divergence-free condition $\nabla \cdot \uu = 0$).  This gives (after some algebra)
\begin{align}
\nabla \uu & = \frac{1}{r} \mx{-xy\gamma' & -y^2\gamma'-r\gamma \\ x^2\gamma'+r\gamma & xy\gamma'}, \\
\ta & = \frac{\mu \gamma'}{r} \mx{-2xy & x^2-y^2 \\ x^2-y^2 & 2xy}, \\
\nabla \cdot \ta & = \frac{\mu}{r} \left( r \gamma'' + 3 \gamma' \right) \mx{-y \\ x}.
\end{align}
Rotational symmetry further dictates that $p = p(r)$ depends only on $r$ as well, and hence $\nabla p = \frac{p_r}{r} \mx{x \\ y}$ (where $p_r$ denotes the $r$-derivative of $p$).  Assuming zero gravity ($\gg = 0$) together with the steady-state assumption ($\uu_t = 0$) reduces the Navier-Stokes PDE to
$$ \left( \frac{p_r}{\rho r} - \gamma^2 \right) \mx{x \\ y} = \frac{\mu}{\rho r} \left( r \gamma'' + 3 \gamma' \right) \mx{-y \\ x}. $$
Since the LHS and RHS are orthogonal vectors, each must be identically zero, hence
$$ p_r = \rho r \gamma^2 \q r \gamma'' + 3 \gamma' = 0. $$
The latter ODE in $\gamma$ easily solves to $\gamma(r) = \alpha + \beta r^{-2}$ for some constants $\alpha$ and $\beta$, which then gives
$$ p(r) = \frac{\rho}{2} \left( \alpha^2 r^2 + 4 \alpha \beta \log r - \beta^2 r^{-2} \right) $$
plus an additive constant.

Let us now consider the jump conditions on $\uu$ and $p$ for an interface $\Gamma$ corresponding to a circle of radius $r_I$ centered at the origin.  Thus, $\nn$ and $\tt$ are simply $\mx{x \\ y}$ and $\mx{-y \\ x}$, respectively, along $\Gamma$, and the curvature of $\Gamma$ is a constant $1/r_I$.  Some more algebra yields
$$ \left[ \mx{\nn & \tt}^T \left( p\II - \ta \right) \nn \right] = \mx{[p] \\ r_I [\mu \gamma']}, $$
and hence the jump conditions manifest themselves as $[p] = \sigma/r_I$ and $[\mu \beta] = 0$.

Let us now consider boundary conditions.  Suppose our domain is simply an annulus (with no interface) with inner and outer radii $r_a$ and $r_b$, respectively; and let the velocities at the inner and outer boundaries be given by $\frac{u_a}{r_a} \mx{-y \\ x}$ and $\frac{u_b}{r_b} \mx{-y \\ x}$, respectively.  This gives the following system for $\alpha$ and $\beta$:
$$ r_a^2 \alpha + \beta = r_a u_a \q r_b^2 \alpha + \beta = r_b u_b $$
which solves to
$$ \alpha = \frac{r_b u_b - r_a u_a}{r_b^2 - r_a^2} \q \beta = \frac{r_a^{-1} u_a - r_b^{-1} u_b}{r_a^{-2} - r_b^{-2}}. $$

Let us now reintroduce a circular interface at radius $r^I$, and suppose the velocity on the interface is given by $\frac{u^I_0}{r^I} \mx{-y \\ x}$.  Putting the boundary conditions and jump conditions together yields
$$ \alpha^- = \frac{r^I u^I_0 - r^- u^-_0}{(r^I)^2 - (r^-)^2} \q \beta^- = \frac{(r^-)^{-1} u^-_0 - (r^I)^{-1} u^I_0}{(r^-)^{-2} - (r^I)^{-2}} $$
$$ \alpha^+ = \frac{r^+ u^+_0 - r^I u^I_0}{(r^+)^2 - (r^I)^2} \q \beta^+ = \frac{(r^I)^{-1} u^I_0 - (r^+)^{-1} u^+_0}{(r^I)^{-2} - (r^+)^{-2}} $$
$$ \mu^+ \beta^+ = \mu^- \beta^- $$
from which one solves for $u^I_0$:
$$ u^I_0 = \frac{(r^I/r^+) w^+ u^+_0 + (r^I/r^-) w^- u^-_0}{w^+ + w^-} \q w^+ = \mu^+ \left( (r^-)^{-2} - (r^I)^{-2} \right) \q w^- = \mu^- \left( (r^I)^{-2} - (r^+)^{-2} \right) $$

\section{Radial Flow}

We now turn to a purely radial flow, where the velocity takes the form $\uu = \gamma \mx{x \\ y}$ and $\gamma = \gamma(r)$ is again a scalar function of $r = \sqrt{x^2 + y^2}$.  This gives
\begin{align}
\nabla \uu & = \frac{1}{r} \mx{x^2 \gamma' + r \gamma & xy \gamma' \\ xy \gamma' & y^2 \gamma' + r \gamma} \\
\ta & = \frac{2\mu}{r} \mx{x^2 \gamma' + r \gamma & xy \gamma' \\ xy \gamma' & y^2 \gamma' + r \gamma} \\
\nabla \cdot \ta & = \frac{2\mu}{r} \left( r \gamma'' + 3 \gamma' \right) \mx{x \\ y} \\
\left( \nabla \uu \right) \uu & = \gamma \left( r \gamma' + \gamma \right) \mx{x \\ y}
\end{align}
The divergence-free condition, $\nabla \cdot \uu = 0$, gives that $r \gamma' + 2 \gamma \equiv 0$, hence $\gamma(r) = \alpha r^{-2}$ for some constant $\alpha$, and from this we find that $\nabla \cdot \ta \equiv 0$.

Rotational symmetry again lets us assume $p$ to be a function only of $r$, reducing the Navier-Stokes PDE to ($\gg = 0$ and $\uu_t = 0$)
$$ \left( \gamma \left( r \gamma' + \gamma \right) + \frac{p_r}{\rho r} \right) \mx{x \\ y} = 0. $$
Using $\gamma(r) = \alpha r^{-2}$, we find that $p(r) = -\frac{1}{2} \alpha^2 \rho r^{-2}$ plus an additive constant.

The jump condition reduces to
$$ \left[ r p + 2 \alpha \mu \right] = \sigma. $$

\end{document}

