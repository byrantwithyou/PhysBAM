\documentclass{article}
\usepackage{fullpage}
\usepackage{amsmath}

\newcommand{\q}{\quad\quad}
\newcommand{\gp}[1]{{\left({#1}\right)}}
\newcommand{\z}{\mathbf{0}}
\newcommand{\tr}{\mbox{tr}}
\newcommand{\uu}{\mathbf{u}}
\newcommand{\II}{\mathbf{I}}
\renewcommand{\gg}{\mathbf{g}}
\renewcommand{\uu}{\mathbf{u}}
\newcommand{\nn}{\mathbf{n}}
\renewcommand{\tt}{\mathbf{t}}
%\newcommand{\ta}{\mbox{\boldmath{\ensuremath{\tau}}}}
\newcommand{\ta}{\pmb{\tau}}
\newcommand{\mx}[1]{\begin{pmatrix} #1 \end{pmatrix}}
\newcommand{\px}[2]{\frac{\partial #1}{\partial #2}}

\begin{document}

\section{Navier-Stokes Equations}

$$ \rho \uu_t + \rho (\uu \cdot \nabla) \uu + \nabla p = \nabla\cdot \ta + \rho \gg $$

$$ \ta = \mu (\nabla \uu + \nabla \uu^T) $$

$$ \left[ \mx{\nn & \tt}^T (p \II - \ta) \nn \right] = \mx{\sigma \kappa \\ 0} $$

$$ \nabla \cdot \uu = 0 $$

\section{Poiseuille Flow}

Poisseuille Flow describes steady state flow through a long, narrow channel.  We assume no-slip boundary conditions on the channel walls, with the additional assumption that the channel
has infinite length. We assume the flow to be parallel to the channel and laminar, so that $u = 0$.  We also assume translational symmetry along the channel, so that $\px{v}{y}=0$.
Since we will introduce an interface with a jump discontinuity across it, it will be convenient to treat boundary conditions later.  With these assumptions, we have
$$ \uu = \mx{0 \\ v} \q \nabla\uu = \mx{0 & 0 \\ v_x & 0} \q \ta = \mu v_x \mx{0 & 1 \\ 1 & 0} \q \uu_t = \z$$
$$ (\uu \cdot \nabla) \uu = \nabla\uu \uu = \mx{0 & 0 \\ v_x & 0} \mx{0 \\ v} = \z \q \nabla \cdot \uu = \tr\mx{0 & 0 \\ v_x & 0} = 0 $$
$$ \nabla p = \mx{p_x \\ p_y} \q \nabla\cdot \ta = \mx{0 \\ \mu v_{xx}} \q \rho \gg = \mx{0 \\ -\rho g} $$
$$ \mx{p_x \\ p_y} = \mx{0 \\ \mu v_{xx} - \rho g} \q p_x = 0 \q \mu v_{xx} = p_y + \rho g $$
$$ \nn = \mx{1 \\ 0} \q \tt = \mx{0 \\ 1} $$
We note that there are many solutions to the last equation, since there could be a pressure gradient forcing the fluid to flow in addition to gravity.  We assume that there is no such
pressure gradient, so that $p_y = 0$.  This implies that pressure is constant, and it leads to
$$ v = \frac{\rho g}{2 \mu} x^2 + a x + b, $$
for some constants $a$ and $b$.

Next, consider that the left wall is at $x = -w$, the interface is at $x = 0$, and the right wall is at $x = w$.  The velocity to the left of the interface is $v^-$ with density $\rho^-$
and viscosity $\mu^-$.  Similarly, $v^+$, $\rho^+$, and $\mu^+$ describe the fluid to the right of the interface.  We assume density and viscosity are piecewise constant.  Assuming that
a velocity of $v^0$ is obtained at the interface, the boundary conditions lead to
$$ v^- = \frac{\rho^- g}{2 \mu^-} x^2 + \gp{\frac{\rho^- g}{2 \mu^-} w + \frac{v^0}{w}} x + v^0 \q v^+ = \frac{\rho^+ g}{2 \mu^+} x^2 - \gp{\frac{\rho^+ g}{2 \mu^+} w + \frac{v^0}{w}} x
+ v^0 $$
To find $v^0$, we need to consider jump conditions.
$$ \mx{\sigma \kappa \\ 0} = \z \q \left[ \mx{\nn & \tt}^T (p \II - \ta) \nn \right] = [p] \mx{1 \\ 0}  - [\mu v_x] \mx{0 \\ 1} \q [p] = 0 \q [\mu v_x] = 0 $$
$$ 0 = [\mu v_x] = \mu^+ \gp{-\frac{\rho^+ g}{2 \mu^+} w - \frac{v^0}{w}} - \mu^- \gp{\frac{\rho^- g}{2 \mu^-} w + \frac{v^0}{w}} = 
-\frac{g w (\rho^+ + \rho^-)}{2} - \frac{v^0 (\mu^+ + \mu^-)}{w} $$
$$ v^0  = -\frac{g w^2 (\rho^+ + \rho^-)}{2 (\mu^+ + \mu^-)} $$

\section{Couette Flow}

Couette Flow describes steady state shearing flow through a long, narrow channel.  We assume no-slip forced boundary conditions on the channel walls, with the additional assumption that
the channel has infinite length. We assume the flow to be parallel to the channel and laminar, so that $u = 0$.  We also assume translational symmetry along the channel, so that
$\px{v}{y}=0$.  Since we will introduce an interface with a jump discontinuity across it, it will be convenient to treat boundary conditions later.  This case differs from Poiseuille
Flow in that $g = 0$, so that $v = a x + b$.  With an interface in the middle of the channel, the new boundary conditions are $v = v_L$ at $x = -w$ and $v=v_R$ at $x = w$.
$$ v^+ = \frac{v_R - v^0}{w} x + v^0 \q v^- = -\frac{v_L - v^0}{w} x + v^0 $$
The jump conditions are the same as before, so
$$ \mu^+ \frac{v_R - v^0}{w} + \mu^- \frac{v_L - v^0}{w} = 0 \q v^0 = \frac{\mu^- v_L + \mu^+ v_R}{\mu^+ + \mu^-} $$

\section{Circular Couette Flow}



\end{document}


